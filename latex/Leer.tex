\documentclass{article}
\usepackage{listings}

\begin{document}

\section*{Arreglo en C++}

Para representar el arreglo \lstinline{miArreglo}, podemos usar un \lstinline{std::vector} de la siguiente manera:

\begin{lstlisting}[language=C++, frame=single, caption={Ejemplo de un arreglo (vector) en C++}, breaklines=true]
#include <iostream>
#include <vector>

int main() {
    int n = 4;
    std::vector<int> miArreglo = {12, 15, 15, 60, 45};

    // Acceder a un elemento (por ejemplo, el primer elemento)
    std::cout << "El primer elemento es: " << miArreglo[0] << std::endl;

    // Iterar sobre el arreglo
    std::cout << "Elementos del arreglo:" << std::endl;
    for (int valor : miArreglo) {
        std::cout << valor << std::endl;
    }

    return 0;
}
\end{lstlisting}

\end{document}
